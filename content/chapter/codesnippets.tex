%!TEX root = ../../main.tex

\chapter{LaTeX Beispielcode}
Dies ist der Text des ersten Kapitels.Nur erwähnte Literaturverweise werden auch im Literaturverzeichnis gedruckt: \cite[S.12 ff]{baumgartner:2002}, \cite[S.1-3]{dreyfus:1980}

Meine erste Fußnote\footnote{Ich bin eine Fußnote} darf auch nicht fehlen. Fußnoten sind dazu da, dass man Begriffe näher erklärt, die aber dem vertrauten Leser wahrscheinlich eh bekannt sind. 

\begin{figure}[h]
\centering
\includegraphics[height=.8\textwidth]{logo.png}
\caption{Das Logo der Musterfirma\footnotemark}
\end{figure}



%\begin{wrapfigure}{r}{.4\textwidth}
%\centering
%\includegraphics[height=.35\textwidth]{logo.png}
%\vspace{-15pt}
%\caption{Das Logo der Musterfirma\footnotemark}
%\end{wrapfigure}
%Quelle muss in Fußnote stehen (da sonst aufgrund eines Fehlers nicht kompiliert
% wird)
\footnotetext{aus \cite{mustermann:2012}}
Looking for the one superhero comic you just have to read. Following the antics and adventures of May Mayday Parker, this Spider-book has everything you could want in a comic--action, laughs, mystery and someone in a Spidey suit. Collects Alias \#1-28, What If. Jessica Jones had Joined the Avengers. In her inaugural arc, Jessicas life immediately becomes expendable when she uncovers the potentially explosive secret of one heros true identity. In her inaugural arc, Jessicas life immediately becomes expendable when she uncovers the potentially explosive secret of one heros true identity.

Manchmal braucht man auch Formeln. LaTeX hat einen sehr guten Formeleditor, der eigentlich selbsterklärend ist. Die Formeln weden automatisch nummeriert, aber man kann immer im Text mit einem Label wie Formel \ref{xyz} Bezug nehmen.

\begin{equation}
t-t_{0}=\sqrt{\frac{l}{g}}\int_{0}^{\varphi}{\frac{d\psi}{\sqrt{1-k^{2}\sin^{2} {\psi}}}} = \sqrt{\frac{l}{g}} F(k,\varphi)
\label{xyz}
\end{equation}

Manchmal braucht man Aufzählungen, die man in einzelnenen Punkten aufführt.
\begin{itemize}
\item Dies ist der erste Punkt, der aufgeführt wird.
\item Dies ist der zweite Punkt, der aufgeführt wird. Manchmal will man auch etwas \textbf{fett} oder \textit{kursiv} oder \textbf{\textit{beides in Kombination}}  drucken.
\item Dies ist der dritte Punkt, der aufgeführt wird.
\end{itemize}

Once upon a time, Jessica Jones was a costumed super-hero, just not a very good one. First, a story where Wolverine and Hulk come together, and then Captain America and Cable meet up. In a city of Marvels, Jessica Jones never found her niche. The classic adventures of Spider-Man from the early days up until the 90s. Looking for the one superhero comic you just have to read. In her inaugural arc, Jessicas life immediately becomes expendable when she uncovers the potentially explosive secret of one heros true identity.

Erste Erwähnung eines Akronyms wird als Fußnote angezeigt. Jede weitere wird
nur verlinkt: \acf{AGPL}. \cite{fsf:2007}

Verweise auf das Glossar: \gls{Glossareintrag}, \glspl{Glossareintrag}

%title wird unter dem Bsp. abgedruckt
%caption wird im Verzeichnis abgedruckt
%label wird zum referenzieren benutzt, muss einzigartig sein.

\begin{lstlisting}[caption=Code-Beispiel, label=Bsp.1]
    public class HelloWorld {
        public static void main (String[] args) {
            // Ausgabe Hello World!
            System.out.println("Hello World!");
        }
    }
    \end{lstlisting}
    
    %language ändert die Sprache. (Wenn nur eine Sprache verwendet wird, kann diese Sprache in einstellungen.tex geändert werden. Standardmäßig Java.)
    \begin{lstlisting}[caption=Python-Code, label=Python-Code, language=Python]
    def quicksort(liste):
    if len(liste) <= 1:
        return liste
    pivotelement = liste.pop()
    links = [element for element in liste if element < pivotelement]
    rechts = [element for element in liste if element >= pivotelement]
    return quicksort(links) + [pivotelement] + quicksort(rechts)
    # Quelle: http://de.wikipedia.org/wiki/Python_(Programmiersprache)
    \end{lstlisting}
    
    \section{lorem ipsum}
    Looking for the one superhero comic you just have to read. Following the antics and adventures of May Mayday Parker, this Spider-book has everything you could want in a comic--action, laughs, mystery and someone in a Spidey suit. Collects Alias \#1-28, What If. Jessica Jones had Joined the Avengers. In her inaugural arc, Jessicas life immediately becomes expendable when she uncovers the potentially explosive secret of one heros true identity. 
    
    Manchmal braucht man auch Tabellen. Ein Beispiel sieht man in Tabelle \ref{tabelle1}, welche mit einem beliebigen Label bezeichnet werden kann. Die Tabelle taucht dann automatisch im Tabellenverzeichnis auf.
    
    \begin{table}[h!]
    \begin{center}
    \begin{tabular}{ | m{5cm} | m{1cm}| m{1cm} | } 
    \hline
    cell1 dummy text dummy text dummy text& cell2 & cell3 \\ 
    \hline
    cell1 dummy text dummy text dummy text & cell5 & cell6 \\ 
    \hline
    cell7 & cell8 & cell9 \\ 
    \hline
    \end{tabular}
    \end{center}
    \caption{Test der Funktion der Tabelle und ihrer Darstellung}
    \label{tabelle1}
    \end{table}
    
    
    Once upon a time, Jessica Jones was a costumed super-hero, just not a very good one. First, a story where Wolverine and Hulk come together, and then Captain America and Cable meet up. In a city of Marvels, Jessica Jones never found her niche. The classic adventures of Spider-Man from the early days up until the 90s. Looking for the one superhero comic you just have to read.
    
    Meet all of Spideys deadly enemies, from the Green Goblin and Doctor Octopus to Venom and Carnage, plus see Peter Parker fall in love, face tragedy and triumph, and learn that with great power comes great responsibility. In a city of Marvels, Jessica Jones never found her niche. Bitten by a radioactive spider, high school student Peter Parker gained the speed, strength and powers of a spider. Looking for the one superhero comic you just have to read. What do you get when you ask the question, What if Spider-Man had a daughter.
    
    The classic adventures of Spider-Man from the early days up until the 90s. Amazing Spider-Man is the cornerstone of the Marvel Universe. But will each partner’s combined strength be enough. Adopting the name Spider-Man, Peter hoped to start a career using his new abilities. Youve found it.
    
    \section{Verweis auf Code}
    Verweis auf den Code \autoref{Bsp.1}.\\
    und der Python-Code \autoref{Python-Code}.
    
    Zweite Erwähnung einer Abkürzung \ac{AGPL} (Erlärung wird nicht mehr angezeigt)

