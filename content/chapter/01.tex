\chapter{Einführung}\label{ch:einfuehrung}
Zwischen 1985 und 1987 wurden sechs Patienten in vier Krankenhäusern in den USA und Kanada von Therac-25 Maschinen zur Strahlentherapie extreme Überdosen an Strahlung verabreicht. Drei Patienten starben, drei weitere wurden schwer verletzt. Der Grund hierfür war nicht etwa menschliches Versagen der behandelnden Ärzte, sondern ein Fehler in der Software der Therac-25 Maschinen. \cite[S. 425]{baase:2008} Schon zum damaligen Zeitpunkt wurde man sich bewusst, welche katastrophalen Auswirkungen Fehler in Software haben können. 

Nicht nur der Therac-25-Fall zeigt, wie tiefgreifend Softwaremängel Schäden sowohl finanzieller als auch persönlicher Art anrichten können. Im Jahr 1999 führte die britische Post Office Ltd. ein System namens \textit{Horizon} vom Japanischen Hersteller Fujitsu ein, welches für die Buchhaltung zuständig war. In den Folgejahren bis ins Jahr 2014 wurden 736 Sub-Postmaster fälschlicherweise der Unterschlagung von Geld teilweise in Höhe von mehreren Tausend Pfund beschuldigt, da \textit{Horizon} Unregelmäßigkeiten in der Buchhaltung meldete. Dies stellte sich erst Jahre später als Fehler der Software heraus - zu diesem Zeitpunkt hatten viele ehemalige Angestellte bereits unschuldig Zeit im Gefängnis verbracht und Existenzen standen aufgrund hoher Schulden in Gefahr oder wurden sogar gänzlich ruiniert. \cite{bbc:2022} Ob Softwarefehler Menschen in einem solchen Ausmaß betreffen oder nicht - viel Geld kosten können sie in jedem Fall: Laut dem \ac{NIST} betrug der Kostenfaktor durch Softwarefehler in den USA bereits im Jahr 2002 59,9 Milliarden US-Dollar pro Jahr. \cite{nist:2002}

Es wird also klar, dass das Gewährleisten von hochqualitativer Software schon in früheren Jahren der Softwareentwicklung von großer Wichtigkeit war. Doch auch heute, trotz rapider Fortschritte der Technologie und trotz über Jahre optimierter Methoden zur Qualitätssicherung, sind Softwarefehler noch immer ein stets präsentes Problem. Der Elektroautohersteller Tesla, bekannt für Innovation und Fortschrittlichkeit, rief innerhalb der letzten Jahre gleich mehrmals Autos zurück, nachdem immer wieder Probleme mit der Software auftraten. Hunderttausende Autos waren und sind teils immer noch von den vielen Fehlern betroffen. \cite{kbb:2022}

Die Geschichte der Softwareentwicklung zeigt, dass Softwarefehler nie völlig eliminiert werden können. Jedoch bietet das ausführliche Testen von Software die Möglichkeit, so viele Fehler wie möglich schon vor der Auslieferung von Software aufzudecken und zu beheben und so die potentiellen Auswirkungen von Bugs zu minimieren. Gerade Testautomatisierung ermöglicht dabei hochfrequentes, effizientes Testen eines Systems, sodass Fehler frühstmöglich entdeckt werden. Doch auch Tests selbst sind nicht fehlerfrei - tatsächlich finden sich auch in Softwaretests eine Vielzahl von Fehlern. \cite{vahabzadeh:2015} Invalide Testdaten können beispielsweise dazu führen, dass ein Test fehlschlägt. So wie das Automatisieren der Tests selbst zu einer Verbesserung der Softwarequalität führen kann, bietet eine Automatisierung des hiermit immer verbundenen Testdatenerstellungsprozesses einen interessanten Ansatz, Softwaretests zuverlässiger zu machen. Dies kann wiederum also auch zum Verbessern der allgemeinen Softwarequalität beitragen. Genau dies, das Gewährleisten von möglichst zuverlässiger und hochwertiger Software und damit verbunden auch die Optimierung des Testprozesses, ist ein wichtiger Aspekt in der Softwareentwicklung bei der Firma \textit{FNT}. 

\section{Die Firma FNT}\label{sec:fnt}
Die \textit{\companyName{}} ist ein Unternehmen aus Ellwangen an der Jagst, welches als Marktführer in diesem Bereich Softwarelösungen für die Verwaltung von IT-, Rechenzentrums- und Telekommunikationsinfrastrukturen für Kunden weltweit entwickelt. \cite{fnt:2021} Das wohl bekannteste Produkt der Firma stellt die Software \textit{FNT Command} dar. Mit dieser Software können Kunden ihre gesamte IT-Infrastruktur übersichtlich an einem Ort erfassen, planen und verwalten, was gerade in Anbetracht der heutigen Dichte solcher Infrastrukturen ein immer wichtigeres Investment für viele Firmen darstellt. Der langjährige Erfolg und die vielversprechenden Zukunftsaussichten der \textit{\companyName{}} zeigen sich nach außen hin gerade durch das kontinuierliche Wachstum der Firma seit ihrer Gründung im Jahr 1994. So zählt das Unternehmen heutzutage mehr als 300 Mitarbeiter*innen in verschiedenen internationalen Standorten.

\section{Problemstellung und Motivation}\label{sec:motivation}
Software, die von \textit{FNT} an Kunden ausgeliefert wird, soll qualitativ hochwertig und zuverlässig sein. Hierfür existiert ein eigener Bereich bei \textit{FNT}, welcher auf \ac{QA} fokussiert ist. In diesem Bereich befasst sich eine \ac{CoP} damit, den Qualitätssicherungsprozess konstant auf einem hohen Niveau zu halten und Richtlinien und Standards für das Testen von Software zu formulieren. Eine \ac{CoP} ist eine Gruppe von Menschen, die ein gemeinsames Problem oder Interesse verfolgen und dabei durch regelmäßigen Austausch versuchen, die Arbeit im betrachteten Feld - in diesem Fall die Qualitätssicherung - nachhaltig zu verbessern. \cite{wenger:2022} Es wird deutlich, dass das ausführliche Testen von Software als wesentlicher Faktor der \ac{QA} bei \textit{FNT} einen hohen Stellenwert besitzt.

In vielerlei Bereichen kommen bei \textit{FNT} daher automatisierte Tests zum Einsatz. Diese bieten den Vorteil, dass bei Übernahme von Änderungen in den Hauptcode direkt automatisch ein Reihe von Tests in einer Pipeline durchgeführt wird und so ohne großen manuellen Testaufwand bestimmt werden kann, ob durch eine Änderung im Code die korrekte Funktionalität der Software beeinträchtigt wurde. Auch für das \ac{CIF} wurde im Jahr 2021 ein Projekt für automatisierte Tests im Rahmen der Bachelorarbeit von Kathrin Ulmer ins Leben gerufen; zuvor wurde dieses noch sehr aufwändig manuell von designierten Tester*innen getestet. \cite{ulmer:2021} Das Testprojekt ist zum Zeitpunkt dieser Arbeit noch nicht fertiggestellt; viele Testfälle sind daher noch nicht implementiert, jedoch konzeptioniert und auf der Testplattform \textit{TestRail} detailliert dokumentiert.

Für alle automatisierten Tests bei \textit{FNT}, also auch insbesondere beim automatisierten Testprojekt für das \ac{CIF}, gilt allerdings, dass Testdaten manuell per Hand erstellt werden müssen und nicht auch wie die Tests automatisiert ablaufen beziehungsweise generiert werden. Dies führt dazu, dass viel Zeit bei der Implementierung und Ausführung der Tests darin einfließt, Testdaten zu formulieren und aufwändig zu überprüfen, ob diese den Anforderungen der einzelnen Testfälle entsprechen. Auch müssen die Testdaten gepflegt werden, da sie statisch in externen Dateien gehalten werden. Somit müssen bei Veränderungen der Beschaffenheit von Objekten in der zu testenden Software manuell eine große Anzahl von Daten über viele Dateien händisch angepasst werden. All dies ist nicht nur aufwändig, sondern auch sehr fehleranfällig, was auch durch eigene Erfahrungen, welche in Kapitel \ref{sec:testimplementierung} dargestellt sind, bestätigt werden konnte. Fehlerhafte Testdaten sind dabei ein ernstzunehmendes Problem, da so die Abdeckung aller Funktionalitäten einer Software durch Tests nicht gewährleistet werden kann. Ein Testfall kann korrekt implementiert, durch unpassende Testdaten allerdings invalidiert werden. \cite[S. 137]{oregan:2019} Im Bereich der Testdatenerstellung zeigt sich also deutliches Optimierungspotential.

\section{Abgrenzung des Themas}\label{sec:abgrenzung}
Im Rahmen dieser Projektarbeit soll das Erstellen von Testdaten für das automatisierte Testprojekt des \ac{CIF} ebenfalls weitestgehend automatisiert werden. Hierbei soll der Fokus auf der Erstellung eines realisierbaren Konzepts liegen und die Umsetzung so weit wie möglich erfolgen, aber keinesfalls einen Anspruch auf Vollständigkeit erheben. Vielmehr soll eine prototypische Realisierung als eine Art \enquote{Proof of Concept} dienen, um so zu zeigen, dass das erarbeitete Konzept auch in der Realität funktional und anwendbar ist. \cite{oed:2021}

\section{Zielsetzung}\label{sec:zielsetzung}
Das Ziel dieser Projektarbeit ist die Konzeptionierung und prototypische Umsetzung einer automatischen Testdatengenerierung für das automatische Testprojekt des \ac{CIF}. Hierbei sollte es nach einer vollständigen Implementierung des Konzepts möglich sein, nahezu alle bisher manuell durchgeführten Schritte in Verbindung mit Testdaten zu automatisieren und somit obsolet zu machen. Die prototypische Umsetzung sollte so geschehen, dass diese bei Bedarf einfach von einem*einer anderen Entwickler*in übernommen und fortgeführt werden kann. Das gesamte Vorgehen sollte ausführlich dokumentiert und die Dokumentation allen berechtigten Beschäftigten von \textit{FNT} zur Verfügung gestellt werden.