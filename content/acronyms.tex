%!TEX root = ../main.tex

\addchap{\acronymsPhrase}

\begin{acronym}[YTMMM]
\setlength{\itemsep}{-\parsep}

\acro{AGPL}{Affero GNU General Public License}
\acro{CIF}{Command Integration Framework}
\acro{SQL}{Structured Query Language}
\acro{NMS}{Network Management System}
\acro{QA}{Quality Assurance (Qualitätssicherung)}
\acro{SUT}{System Under Test}
\acro{GUI}{Graphical User Interface (Grafische Benutzeroberfläche)}
\acro{API}{Application Programming Interface (Programmierschnittstelle)}
\acro{BGE}{Business Gateway Entity}
\acro{OSS}{Open Source Software}
\acro{REST}{Representational State Transfer}
\acro{JSON}{JavaScript Object Notation}
\acro{POJO}{Plain Old Java Object}
\acro{Elid}{Element ID}
\acro{CSV}{Comma Separated Value}
\acro{XML}{Extensible Markup Language}
\acro{CoP}{Community of Practice}
\acro{HTTP}{Hypertext Transfer Protocol}
\acro{E2E}{End-to-End}

\end{acronym}
