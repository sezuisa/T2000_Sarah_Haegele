%!TEX root = ../main.tex

\pagestyle{empty}

% override abstract headline
\renewcommand{\abstractname}{Abstract}

\begin{abstract}
Das ausgiebige Testen ist essentiell zur Entwicklung qualitativ hochwertiger Software. Gerade automatisiertes Testen hilft dabei, Testzyklen häufiger durchlaufen zu können und so kleinere Codeänderungen direkt zu prüfen, sodass Fehler möglichst früh entdeckt werden. Die Flexibilität der automatisierten Tests kommt allerdings an ihre Grenzen, wenn die zugrunde liegenden Testdaten manuell gepflegt werden müssen. Das automatische Generieren dieser Testdaten kann dabei helfen, automatisierte Tests noch flexibler und zuverlässiger zu gestalten.

In dieser Arbeit wird eine automatische Testdatengenerierung für die automatisierten Tests des Command Integration Frameworks, einem System zur Integration externer Massendaten, konzeptioniert und realisiert. Zunächst wird der Ist-Zustand analysiert und dessen Schwächen aufgezeigt. Daraufhin wird ein Konzept zur Umsetzung einer Datengenerierung basierend auf definierten Anforderungen erarbeitet. Dieses wird für einige Anwendungsfälle implementiert und erweitert und anschließend in Hinblick auf die Anforderungen, Ziele und den Implementierungsaufwand evaluiert.

\vspace*{20mm}

Thorough testing is essential to the development of high quality software. Automated testing helps in enabling more frequent test runs so that smaller code changes can be validated immediately and faults can be detected as early as possible. The flexibility of automated tests reaches its limits, however, when the underlying test data has to be maintained manually. Generating test data automatically can help in making automated tests more flexible and reliable.

In this thesis, an automated test data generation for the automated tests of the Command Integration Framework, a system for external mass data integration, is conceptualised and realised. Firstly, the status quo is analysed and its weaknesses are outlined. Secondly, a concept for the realisation of the data generation is created based on defined requirements. This concept is implemented for multiple use cases, extended and finally evaluated in regards to the requirements, goals and the implementation effort.
\end{abstract}